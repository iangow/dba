\documentclass[11pt]{amsart}
\usepackage[marginratio=1:1, margin=0.8in]{geometry}  % See geometry.pdf to learn the layout options. There are lots.
\usepackage{palatino}
\usepackage{natbib}
\usepackage[parfill]{parskip} 

\title[HBS 4403: Management Control and Performance Measurement]{HBS 4403: Problem Set 3}

\author{Ian D. Gow}

\begin{document}
\maketitle

\bibliographystyle{chicago}

\citet{Badolato:2014bw} suggests that the ``financial expertise" of the audit committee reduces earnings management. 
This paper builds on a large literature examining financial expertise of directors.
For example, \citet{DeFond:2005hs} finds a significant positive market reaction to the appointment of directors with financial expertise, \citet{Dhaliwal:2010bz} finds that the financial expertise of the audit committee is associated with higher accrual quality, and \citet{Engel:2010kv} finds that directors with financial expertise are paid more.

\section{Assignment Details}
You conjecture that the financial expertise of the board mitigates misreporting. 
Your assignment is to test this conjecture using the supplied data. 
You may test any hypothesis related to this conjecture---assume there is no prior literature on financial expertise of directors, and that if published, your paper will be the first study on this topic.
You may measure misreporting and financial expertise however you like; the only constraint is that you cannot use data beyond what is supplied. 
The supplied data will enable you to choose from a variety of measures of misreporting and financial expertise. 
You may also choose whatever variables (from the supplied data) you like for your analysis. 
You may construct new variables and apply transformations to existing variables as you see fit (e.g., quintiles, deciles, logs, absolute values, indicator variables, fixed effects, industry adjustments, changes, interactions, ratios). 

Of course you may ask your peers (or me) for help, but you should not work together.

\section{Document outline}
Prepare a document in which you develop a single test of your conjecture. 
Articulate the rationale behind your test and measurement choices, and discuss and tabulate your results. 
The document should include (perhaps as an appendix) source code (in either R or Stata) that will allow a reader to replicate your analysis using the supplied data.\footnote{
Email me if you want to use something other than R or Stata.}
The document should include the sections outlined below.

\subsection{Research design}
Carefully motivate and explain the design of your test. 
Enumerate and justify each choice that you make regarding  variable measurement and research design. 

\subsection{Results}
\begin{itemize}
    \item A detailed description of the variables used in your analysis. 
    \item A simple table that clearly presents your results. Descriptive statistics are not necessary.
    \item Discuss of results as they relate to your conjecture.
\end{itemize}

\section{In-class discussion}
Copies of the document you prepare (as well as the source code you used) should be emailed to me on Sunday, April 19.
In the class  Monday, April 20, you will discuss your test and findings (no slides or other materials are necessary).     

\section{Data}
The data set comprises 10,000 firm-years from 1992 to 2010. 
Most of the data come from \citet{Armstrong:2013hs}. 
Let me know if you have any questions.


\begin{table}[h]
\caption{Variable list} \label{tab:vars}
    \begin{tabular}{ll}
    \textbf{Variable} & \textbf{Description } \\
    \hline
        firmid        & Unique firm identifier \\
        fyear         & Fiscal year \\
        ffic12        & Fama-French 12-industry group  \\
        ffic48        & Fama-French 48-industry group \\
        aaer          & Indicates that an SEC AAER found accounting fraud or  \\
                      & misrepresentation occurred at the firm during the year \\
        restate       & Indicates financial results for that year were restated \\
        restate\_hlm  & Indicates that financial results for that year were restated due \\
                      & to fraud, misrepresentation, or investigation by SEC or PCAOB \\
        dacc\_mj      & Discretionary accruals from the modified \citet{Jones:1991ib} model \\
        dacc\_dd      & Discretionary accruals from \cite{Dechow:2002} model \\
        dacc\_m       & Discretionary accruals from the \citet{McNichols:2002} model \\
        dacc\_lr      & Discretionary accruals from the \citet{Larcker:2004ft} model \\
        num\_meet     & Number of board meetings during year \\
        num\_dir      & Number of directors on the board \\
        num\_dir\_aud & Number of directors on the audit committee \\
        num\_exp      & Number of financial experts on the board \\
        num\_exp\_aud & Number of financial experts on the audit committee 
\end{tabular}    
\end{table}


\bibliography{dba}

\end{document}
